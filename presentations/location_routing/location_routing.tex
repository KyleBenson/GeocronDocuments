\documentclass{beamer}
\usetheme{Frankfurt}


\begin{document}


\AtBeginSection[]
{
   \begin{frame}
        \frametitle{Roadmap}
        \tableofcontents[currentsection,currentsubsection]
   \end{frame}
}

\title[Location routing]{Location-Based Routing}
\subtitle{An overview and possible directions for GeoCRON}
\author[K. Benson]{Kyle E. Benson}
\institute[UCI]{
  Department of Computer Science\\
  University of California, Irvine\\
  Irvine, California 92697\\[1ex]
  \texttt{kebenson@uci.edu}
}


\begin{frame}[plain]
	\titlepage
\end{frame}

% Breaking intro into its own part suppresses navigation bar info
\part{intro}

\begin{frame}{Introduction}

\begin{itemize}
	\item Traditional routing
	\begin{itemize}
		\item Unique address: IP, MAC, Peer ID, etc.
		\item Source routing: next hop address, neighbor index
		\item Local routing: distance-vector, link state, label-switching
	\end{itemize}
	
	\pause
	\item Why location information?
	\begin{itemize}
		\item Geocast: message all (or some) nodes in target region
		\item Latency: request from closer server, route locally when possible
		\item Congestion: confine route requests to smaller regions (MANETs)
		\item Energy: closer nodes need less radio power to reach
		\item Sensors: regional event detection, spatial querying
		\item Planning: paths (robots), surveillance cameras (focus on area target will appear next)
		\alert<3>{\item Recovery: avoid problematic areas of the network}
		\begin{itemize}
		\alert<3>{\uncover<3>{\item our primary interest!}}
		\end{itemize}
	\end{itemize}
\end{itemize}

\end{frame}

% % % % % % % % % % % % % % % % % % % % % % % % % % % % % % % % % % % % % % % % % %
\part{rest}

\section{Location Service}

\begin{frame}{Location Service}

	\begin{itemize}
		\item Nodes have GPS
		\item But how to look up destination's location?
		
		\begin{itemize}
			\item placeholder
		\end{itemize}
		
	\end{itemize}

\end{frame}

% % % % % % % % % % % % % % % % % % % % % % % % % % % % % % % % % % % % % % % % % %

\section{Greedy Forwarding}

\begin{frame}{Location Service}
mention voids
\end{frame}

% % % % % % % % % % % % % % % % % % % % % % % % % % % % % % % % % % % % % % % % % %

\section{Trajectory Routing}

% % % % % % % % % % % % % % % % % % % % % % % % % % % % % % % % % % % % % % % % % %

\section{Geometric Routing}

\begin{frame}{Geometric Routing}
right-hand rule analagous to following the right hand wall in a maze
introduced in Compass Routing on Geometric Networks
\end{frame}

% % % % % % % % % % % % % % % % % % % % % % % % % % % % % % % % % % % % % % % % % %

\section{Clustering}

\begin{frame}{Clustering}
In \cite{779923}, an inter-zone clustering protocol is periodically run to update with information about inter-zone links.
Does not give information about exact location of destination within a zone and so still need to find that.
\end{frame}

\part{bib}

\begin{frame}{References}
\bibliographystyle{IEEEtran}
% argument is your BibTeX string definitions and bibliography database(s)
\bibliography{IEEEabrv,location_routing}
\end{frame}


\end{document}
