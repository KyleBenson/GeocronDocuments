\documentclass{beamer}
\usetheme{Boadilla}

\begin{document}

\title[Location routing]{Location-Based Routing}
\subtitle{An overview and possible directions for GeoCRON}
\author[K. Benson]{Kyle E. Benson}
\institute[UCI]{
  Department of Computer Science\\
  University of California, Irvine\\
  Irvine, California 92697\\[1ex]
  \texttt{kebenson@uci.edu}
}


\begin{frame}
	\titlepage
\end{frame}


\begin{frame}{Introduction}

\begin{itemize}
	\item Traditional routing
	\begin{itemize}
		\item Unique address: IP, MAC, Peer ID, etc.
		\item Source routing: next hop address, neighbor index
		\item Local routing: distance-vector, link state, label-switching
	\end{itemize}
	
	\pause
	\item Why location information?
	\begin{itemize}
		\item Geocast: deliver messages to all (or some) nodes in target region
		\item Latency: request from closer server, route locally when possible
		\item Congestion: confine route requests to smaller regions (MANETs)
		\item Energy: closer nodes need less radio power to reach
		\item Sensors: regional event detection, spatial querying
		\item Planning: paths (robots), surveillance cameras (focus on area target will appear next)
		\alert<3>{\item Recovery: avoid problematic areas of the network}
	\end{itemize}
\end{itemize}

\end{frame}


\begin{frame}{Overview}
	\begin{itemize}
		\item Location service
		\item Greedy
		\item Geometric
		\item Clustering
	\end{itemize}
\end{frame}


\begin{frame}{Location Service}
mention voids
\end{frame}


\begin{frame}{Geometric Routing}
right-hand rule analagous to following the right hand wall in a maze
introduced in Compass Routing on Geometric Networks
\end{frame}

\end{document}
