\documentclass[pdftex]{beamer}
%\usetheme{Frankfurt}

% declare the path(s) where your graphic files are
% ../.. is the GeocronDocuments directory
\graphicspath{{../../images/diagrams/}}
\DeclareGraphicsExtensions{.pdf,.png}

\begin{document}

\title[Short Title]{Simulating Disaster Scenarios and Geographically-Correlated Resilient Overlay Networks}
\subtitle{Heuristics for Location-based Routing}
\author[K. Benson and Z. Huang]{Kyle E. Benson and Zhipeng Huang}
\institute[UCI]{
  Department of Computer Science\\
  University of California, Irvine\\
  Irvine, California 92697\\[1ex]
  \texttt{kebenson@uci.edu} and \texttt{zhipengh@uci.edu}
}

% % % % % % % % % % % % % % % % % % % % % % % % % % % % % % % % % % % % % % % % % %

\begin{frame}[plain]
	\titlepage
\end{frame}

% % % % % % % % % % % % % % % % % % % % % % % % % % % % % % % % % % % % % % % % % %

\begin{frame}{Orthogonal Distant Path Heuristic}
\begin{columns}
\begin{column}{.5\textwidth}

% re-works this itemize into a different background slide.
% put algorithm here instead, maybe an observation/intuition

\begin{itemize}
	\item Failure along path to server or in local area
	\item Choose node outside local	region to avoid overlapping	paths
	\item Choose path avoiding as much of the direct path as possible
	\item Overlay node may use similar route to sensor
\end{itemize}
\end{column}
	
\begin{column}{.5\textwidth}
\includegraphics[height=\textwidth,angle=-90]{angular_path}
\end{column}


% % % % % % % % % % % % % % % % % % % % % % % % % % % % % % % % % % % % % % % % % %

\begin{frame}{Orthogonal Distant Path Heuristic}
\begin{columns}
\begin{column}{.5\textwidth}

% re-works this itemize into a different background slide.
% put algorithm here instead, maybe an observation/intuition

\begin{itemize}
	\item Failure along path to server or in local area
	\item Choose node outside local	region to avoid overlapping	paths
	\item Choose path avoiding as much of the direct path as possible
	\item Overlay node may use similar route to sensor
\end{itemize}
\end{column}
	
\begin{column}{.5\textwidth}
\includegraphics[height=\textwidth,angle=-90]{angular_path}
\end{column}
H1 routes around failed R1
using H3 as an overlay
 After a failure, H1 tries
alternates until success or max
# retries reached
 One overlay hop usually
enough [Han2005]
– For multi-hop, search
space = O(n!/(n-k)!),
where n = # overlay
nodes and k = # hops



\end{columns}
\end{frame}


% % % % % % % % % % % % % % % % % % % % % % % % % % % % % % % % % % % % % % % % % %


% % % % % % % % % % % % % % % % % % % % % % % % % % % % % % % % % % % % % % % % % %

%
%\part{rest}
%
%\begin{frame}{Roadmap}
%	\tableofcontents
%\end{frame}
%
%\section{Location Service}
%
%\begin{frame}{Location Service}
%\begin{columns}
%\begin{column}{.5\textwidth}
%
%	\begin{itemize}
%		\item Nodes have GPS
%		\item But how to look up destination's location?
%		\item Maintain global information \uncover<2->{\alert{easily outdated/inefficient}}
%		\item<3-> Distribute load
%		\begin{itemize}
%			\item In , node updates \emph{location servers} (LS) throughout network
%			\item Divide network into hierarchical grid
%			\item LS's in 3 external grids at each level
%			\item Lookup distance $<$ square LS co-resides in
%		\end{itemize}
%		
%	\end{itemize}
%
%\end{column}
%
%\begin{column}{.5\textwidth}
%\uncover<3->{
%\begin{figure}
%\includegraphics[width=\textwidth]{location_service}
%\caption{Hierarchical grid with 4 order-i squares in order-i+1 square.}
%\end{figure}}
%\end{column}
%\end{columns}
%\end{frame}
%

\end{document}
